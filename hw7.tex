%% HOW TO USE THIS TEMPLATE: %%
%% Type your solution to each problem part within
%% the \begin{solution} \end{solution} environment immediately
%% following it.  Use any of the macros or notation from the
%% header.tex that you need, or use your own (but try to stay
%% consistent with the notation used in the problem).
%%
%% If you have problems compiling this file, you may lack the
%% Header.tex file (available on the course box page), or your system
%% may lack some LaTeX packages.  The "exam" package (required) is
%% available at:
%%
%% http://mirror.ctan.org/macros/latex/contrib/exam/exam.cls
%%
%% Other packages can be found at ctan.org, or you may just comment
%% them out (only the exam and ams* packages are absolutely required).


% The "answers" option causes the solutions to be printed.
% The pdf for this homework that's on blackboard was compiled without the
% answers flag for compactness. You need to include it.
%\documentclass[11pt,addpoints]{exam}
\documentclass[11pt,addpoints,answers]{exam}

% required macros -- get latest hwheader.tex file from blackboard if compiling
\input{hwheader}

\newcommand{\TSAT}{\prob{3-SAT} }
\newcommand{\ILP}{\prob{Integer\ Linear\ Programming} }
\newcommand{\IS}{\prob{Independent\ Set} }
\newcommand{\VC}{\prob{Vertex\ Cover} }
\newcommand{\CQ}{\prob{Clique} }
\newcommand{\HC}{\prob{Hamiltonian\ Cycle} }
\newcommand{\HP}{\prob{Hamiltonian\ Path} }
\newcommand{\KS}{\prob{Knapsack} }
\newcommand{\TSP}{\prob{Traveling\ Salesman\ Problem} }
\newcommand{\SSM}{\prob{Subset\ Sum} }
\newcommand{\SCov}{\prob{Set\ Cover} }

\hwheader{Homework 7}
{YOUR NAME HERE}
{Spring 2017, Bhakta}
{CS 315, Algorithms}
{Due: 3:00pm Friday, Dec 8}

% VARIABLES

\begin{document}

\pagestyle{head}                % put header on every page

\noindent 
{\bf Directions for Homework 7}: You may assume from class that the following
problems (both the search and decision versions) are NP-complete when doing
your reductions. We will (eventually) cover the reasons that all these
problems are NP-complete in class.
\begin{itemize}
  \item \SAT: Given a boolean formula $F$ in CNF form with $n$ variables and
    $m$ clauses, determine if there is an assignment of the $n$ variables that
    satisfies $F$.
  \item \TSAT: A special case of \SAT where every clause has at most 3
    literals.
  \item \ILP: Given a set of $n$ variables and $m$ linear constraints, and a
    linear objective function, find an \emph{integer} solution to the
    constraints that optimizes the objective.
  \item \IS: Given graph $G$ and $k\in\Z^+$, determine if $G$ has an
    indep. set of size $\geq k$.
  \item \VC: Given graph $G$ and $k\in\Z^+$, determine if $G$ has a vertex
    cover of size $\leq k.$
  \item \CQ: Given graph $G$ and $k\in\Z^+$, determine if $G$ has a clique
    of size $\geq k$.
  \item \SSM: Given a list of integers $A$ and a target $T$, determine if
    some sublist of $A$ adds up to $T$.
  \item \KS: Given a list of $n$ items with weights $w_i$ and values $v_i$
    a maximum capacity $C$ and a target value $V$, determine if some subset of
    of the items have a total weight $\leq C$ and a total value $\geq V$.
  \item \HP(Rudrata):  Given graph $G$, determine if $G$ has a path that
    visits every vertex exactly once.
  \item \HC(Rudrata):  Given graph $G$, determine if $G$ has a cycle that
    visits every vertex exactly once.
\end{itemize}

\begin{questions}
  \question[20]
  We define the problem $\CQ3(G,k)$.  You are tasked with determining if $G$
  has a clique of size $k$, but only on speical inputs where the max degree of
  any vertex of $G$ is $3$. In otherwords, this is a special case of the
  $\CQ(G,k)$ problem, but we are restricted to graphs in which every vertex
  has degree at most 3.
  \begin{parts}
    \part Prove that $\CQ3$ is in NP.
    \begin{solution}
      %Tour Solution Goes Here
    \end{solution}

    \part I claim that $\CQ3$ is NP-complete.
    \begin{proof}
      We showed that $\CQ3$ is in NP in part a.

      We show a reduction from $\CQ3$ to \CQ, a known NP-hard problem.
      Clique is a generalizaion of $\CQ3$, since $\CQ3$ consists of the
      special case of \CQ where the input graph has max degree $3$.
      Therefore we can solve $\CQ3(G,k)$ by returning $\CQ(G,k)$ on exactly the
      input to $\CQ3$ unchanged.
      
      This proves the correctness of the reduction and, therefore the
      NP-completeness of Clique-3.
    \end{proof}

    What is wrong with the above proof?

    \begin{solution}
      %Tour Solution Goes Here
    \end{solution}

    \part Here's a true fact that you can use without proof.

    \begin{theorem}
      $\IS3(G,k)$, the special case of $\IS(G,k)$ restricted to graphs of
      degree at most 3, is NP-complete.
    \end{theorem}

    Here's another fact that we learned in lecture when showing that \CQ
    was NP-complete.
    \begin{theorem}
      A subset $X \subseteq V$ is an independent set in $G$ if and only if
      $X$ is a clique in the ``opposite'' graph $G' = (V,E' = {v \choose 2}
      \setminus E)$.
      Therefore $G$ has a independent set of size $\geq k$ if and only $G'$
      has a clique of size $\geq k$.
    \end{theorem}

    Now lets try again.

    Claim: $\CQ3$ is NP-complete.
    
    \begin{proof}
      $\CQ3$ is in NP as proved in part a.

      We present a reduction from $\IS3$ to $\CQ3$. 
      
      To solve $\IS3(G,k)$, we first create $G'$ by switching all the edges of
      $G$, and returning $\CQ3(G',k)$.

      Proof of correctness:
      By the second fact, $G$ has an independent
      set of size $k$ if and only if $G'$ has a clique of size $k$. This
      proves the correctness of the reduction.
      
      By the first fact, $\IS3$ is NP-complete, and therefore $\CQ3$ is
      NP-complete.
    \end{proof}

    What's wrong with the proof this time?
    \begin{solution}
      %Tour Solution Goes Here
    \end{solution}

    \part Describe any poly-time algorithm for $\CQ3$. Hint: How big can a
    clique get?
    \begin{solution}
      %Tour Solution Goes Here
    \end{solution}

  \end{parts}

  \question[40] Generalizations:
  Show that each of the following problems are NP-hard. (Don't worry about
  showing that these problems are in NP). Each reduction should be ``simple'' -
  each problem below is either a direct generalization of a known
  NP-hard problem, or solves a known NP-hard problem with a relatively simple
  transformation (like from Independent Set to Clique).
  \begin{parts}
    \part (EXAMPLE) \prob{Weighted\ Independent\ Set}: Given a graph $G =
    (V,E)$ with integer weights on the vertices $V$, and number $k$, determine
    if $G$ has an independent set of total weight $\geq k$.\\ Prove that this
    problem is NP-complete.
    \begin{solution}
      \begin{itemize}
        \item First we show this is in NP. Say we are given the inputs $G,K$ as
          well as an independent set with total weight $\geq k$. Our verifier
          will check that
          \begin{itemize}
            \item The given vertices have total weight $\geq k$ in at most
              $O(n)$ time.
            \item That this is a valid independent set, by checking each edge
              and making sure that both endpoints are never in the set. This
              will take $O(E)$ time.
            \item Accept if both of the above are True
          \end{itemize}
          This will accept iff the certificate is a valid independent set of
          total weight $\geq k$, which exists iff there is an independent set of
          total weight $\geq k$.
        \item 
          We show that this problem is a generalization of \IS.

          Our reduction: Given an instance $(G,k)$ of $\IS$, create a weighted
          version of $G$ where every vertex has weight $1$. Call this graph
          $G'$. Now, return $\prob{Weighted\ Independent\ Set}(G',k)$.
          \begin{proof}
            By construction, any set of $k$ vertices in $G$ has total weight $k$
            in $G'$ and vice versa. Thus an independent set of $k$ vertices
            exists in $G$ iff that independent set has total weight $k$ in $G'$.
          \end{proof}
      \end{itemize}

    \end{solution}



    \part (EXAMPLE) \prob{SAT_{\leq k}}:
    Given a SAT instance with $n$ variables and $m$ clauses, and an integer $k
    \leq n,$ decide if there is a satisfying assignment in which \emph{at most}
    $k$ of the $n$ variables are assigned TRUE.\\
    Prove that this problem is NP-complete.
    \begin{solution}
      \begin{itemize}
        \item First we show this is in NP. Say we are given the input formula
          $f$, the integer $k$, and a satisfying assignment that sets at most
          $k$ variables ot true. Our verifier will check that
          \begin{itemize}
            \item At most $k$ of the variables are set to true. We can check
              this in $O(n)$ time.
            \item That this is a satisfying assignment by evaluating the
              formula. Each clause takes $O(n)$ time to check as there are at
              most $n$ variables per clause. There are $m$ total clauses, so
              this takes a total of in $O(mn)$ time.
            \item Accept if both of the above are True
          \end{itemize}
          This will accept iff there is a valid assignment
        \item 
          We show that this problem is a generalization of \SAT.

          Our reduction: On an input $F$ to \SAT, we return $\prob{SAT_{\leq
          n}}(F)$.
          
          \begin{proof}
          Since the restriction ``at most $n$ of the variables are set to True''
          applies to any assignment, the problem $\SAT_{\leq n}$ is exactly the
          problem of $\SAT(F)$, and $\SAT$ is a special case of $\SAT_{\leq n}$.
          \end{proof}
      \end{itemize}
    \end{solution}

    \part Given a set of $n$ elements $E$, and a family of
    $m$ subsets $S_i \subseteq E$, and ``budget'' $b$, determine if there are
    some $b$ of these subsets $\set{S_i}$ such that their union is $E$.
    \begin{solution}
      %Your Solution Goes Here
    \end{solution}

    \part \prob{Longest\ s-t\ Path }:
    Given a graph $G$, two vertices $s$ and $t$ in $V$, and an integer $k$
    determine if $G$ has a simple path (no repeat vertices) that uses $k$ or
    more edges from $s$ to $t$.
    \begin{solution}
      %Your Solution Goes Here
    \end{solution}

    \part \TSP: Given a weighted graph $G$ and bound $B$,
    determine if there is a cycle in $G$ that visits every vertex exactly once
    and has the total sum of edge weights $\leq B$.
    \begin{solution}
      %Your Solution Goes Here
    \end{solution}

    \part \prob{Degree\ Restricted\ MST }: Given an undirected graph $G =
    (V,E)$ and an integer $k$, find a spanning tree $T$ of $G$ such that each
    vertex of the tree has maximum degree $k,$ if such a tree exists.
    \begin{solution}
      %Your Solution Goes Here
    \end{solution}

  \end{parts}

  \question (EXAMPLE) \HP: 
  Show that the following problem is NP-complete by reducing from \HC.

  \HP: Given graph $G$, determine if $G$ has a simple {\bf path} (no repeated
  vertices) that visits every vertex exactly once. Here the start and end
  vertices of the path do not need to be the same (or adjacent).
  \begin{solution}
    Note: This solution explains more things than you need to in your
    solutions. This is meant to be instructive.

    \begin{itemize}
      \item To show this problem is in NP, say we are given an input which is
        the graph $G$, as well as the vertices of the path (listed in order).
        Our verifier would check that every vertex was contained in this path,
        and that every proposed edge between adjacent vertices was in the
        graph. If both of these conditions are true, then this is a
        Hamiltonian Path.

      \item 

        Here's our reduction.

        Remember - we are trying to find an algorithm that can efficiently
        solve \HC, a known NP=hard problem, under the assumption that we can
        run an algorithm that can solve \HP, the problem that we are trying to
        show is NP-hard.

        Here's the algorithm:
        Given graph $G$, we're going to construct $G'$ as follows. Pick any
        vertex $v$, and make an extra copy of it called $v'$, and add all the
        same edge connections as $v$.  Now construct another new vertex $s$
        and connect it only to $v$, and another new vertex $t$, and connect it
        only to $v'$. Now, we return $\HP(G')$.

        Here's the proof of correctness of this reduction. We need to show
        that this is CORRECT - that this reduction is an
        algorithm that actually solves \HC. 
        If the input $G$ does have a \HC, we need to return YES. If the input
        $G$ does not have a \HC, we need to return NO.
        
        In this case, that means that showing that $G$ has a \HC if and only
        if $G'$ has an \HP. Here's the proof:

        \begin{itemize}
          \item $G$ has a \HC $\to$ $G'$ has a \HP

            \begin{proof}
              Say $G$ has a \HC. Then since a cycle visits every vertex,
              including $v$, write the vertices of the cycle as $(v, x_1), (x_1,
              x_2), \ldots, (x_{n-1}, v)$. By construction, the edges $(v, x_1),
              (x_1, x_2), \ldots, (x_{n-1}, v')$ are in the graph $G'$, as are
              $(s,v)$ and $(v', t)$. Since the only edges we added were $v', s,
              t$, this new set of edges are a $\HP$ in $G'$.
            \end{proof}

          \item $G'$ has a \HP $\to$ $G$ has a \HC

            \begin{proof}
            Say $G'$ has a \HP.
            Because $s$ and $t$ both have degree 1, any \HP in $G'$ must have
            $s$ and $t$ as its endpoints. Since $s$ only connects to $v$ and
            $t$ to $v'$, then the path must go from $s$ to $v$ to every other
            vertex to $v'$ to $t$. The middle part of this path is a path from
            $v$ to every other vertex back to $v'$ in $G'$. The equivalent
            edges in $G$ would form a path from $v$ to every other vertex and
            back to $v$ again, which are a \HC.
            \end{proof}
        \end{itemize}
    \end{itemize}

    Another valid approach is to add $t$ to a different neighbor of $v$, one
    at a time, until you find a \HP. This would require at most $O(n)$ calls
    to \HP, once for each neighbor of $v$, which is a valid polytime
    reduction.
  \end{solution}

  \question[20] Search \SAT $\to$ Decision \SAT. 
  Describe an algorithm that can {\bf find} a satisfying assignment for a
  given SAT instance (search), if one exists, using an algorithm that can only
  determine {\bf whether or not} a SAT instance \emph{has} a satisfying
  assignment (decision).

  Remember, you may do polynomial work, and call the Decision version of the
  algorithm polynomially many times to solve the Search version.

  Hint: If there is a valid solution to the \SAT instance - use your algorithm
  for Decision \SAT to determine a valid assignment for each variable, one at
  a time.
  \begin{solution}
  \end{solution}

  
  \question[20] \prob{Partition }:
  Show that the following problem is NP-complete by reducing from \SSM.

  Given a list of integers $A$, determine if the integers of $A$ can be
  partitioned into two disjoint lists $A_1, A_2$ that have the same sum.

  For example, the list $A = [7, 3, 2, 9, 6, 13]$ can be partitioned into
  $[7,13]$, $[2,3,6,9]$ each with sum $20$. The list $A = [4,6,8,12]$ cannot be
  partitioned into two equal pieces.
  \begin{solution}
    %Your Solution Goes Here
  \end{solution}

  \question (Bonus) Consider the following problem.
  You are a hiring manager at a consulting firm.
  You have a set of $n$ available workers $W$ and a set of $m$ tasks $T$ that
  your clients want you to do.
  Each worker $w$ knows how to complete some subset $T[w] \subseteq T$ of the
  tasks, and requires a certain salary to be hired, $S[w]$.
  Each worker, once hired, can complete all the tasks that they know
  how to do. Each completed task earns the company a revenue $R[t]$ per task
  $t$.

  You goal is to maximize your profit, the sum of all revenues earned minus
  the sum of all salaries paid.  Not all tasks have to be assigned; sometimes
  the revenue for completing some tasks may not be worth hiring a worker to do
  it. 

  The decision version of this question asks: Given all this information, $n,
  m, T[w], S[w], R[t]$, can you achieve a profit of at least $k$?  Show that
  this problem is NP-complete.
  \begin{solution}
  \end{solution}
\end{questions}
\end{document}


%%% Local Variables: 
%%% mode: latex
%%% TeX-master: t
%%% End: 
